\begin{abstractpage}

\begin{abstract}{romanian}

Procesarea unui volum mare de date este o operațiune computațional costisitoare, o provocare cu care se confruntă atât industria creativă, producția cinematografică sau dezvoltarea de jocuri video, cât și domeniile de cercetare științifică. În primul caz, dificultățile apar în crearea de efecte vizuale impresionante și realiste, iar în al doilea, în simularea unor fenomene fizice avansate. 

Scopul acestui proiect este de a prezenta un sistem de particule real-time modern, performant, optimizat și scalabil, scris în C++ utilizând API-ul grafic Vulkan, care utilizează la maximum capacitățile plăcii video, explorând beneficiile unei arhitecturi în care datele necesare simulării sunt stocate și procesate integral pe GPU. 

Soluția prezentată în această lucrare evidențiază avantajele concentrării sarcinilor de calcul direct pe GPU, punând în lumină modul în care această strategie poate contribui la îmbunătățirea performanței aplicațiilor grafice complexe real-time, unde viteza de procesare și eficiența utilizării resurselor sunt esențiale pentru a obține rezultate cât mai fezabile. 

\end{abstract}

\begin{abstract}{english}

Processing large amounts of data is a computationally intensive task, posing a significant challenge across various fields, including the creative industries, such as film production and video game development, as well as scientific research. In the former case, the primary difficulties arise in generating impressive and realistic visual effects, while in the latter, it stems from the simulation of advanced physical phenomena.

The aim of this project is to present a modern, high-performance, optimized, and scalable real-time particle system, written in C++ using the Vulkan graphics API, which fully leverages GPU capabilities by exploring the benefits of an architecture where all simulation data is stored and processed entirely on the GPU.

The solution presented in this paper highlights the advantages of focusing computational workloads directly on the GPU, highlighting how this approach can significantly improve the performance of complex real-time graphics applications, where processing speed and resource efficiency are essential for achieving the most feasible results.

\end{abstract}

\end{abstractpage}

